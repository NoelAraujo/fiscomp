% Options for packages loaded elsewhere
\PassOptionsToPackage{unicode}{hyperref}
\PassOptionsToPackage{hyphens}{url}
\PassOptionsToPackage{dvipsnames,svgnames,x11names}{xcolor}
%
\documentclass[
  letterpaper,
  DIV=11,
  numbers=noendperiod]{scrreprt}

\usepackage{amsmath,amssymb}
\usepackage{lmodern}
\usepackage{iftex}
\ifPDFTeX
  \usepackage[T1]{fontenc}
  \usepackage[utf8]{inputenc}
  \usepackage{textcomp} % provide euro and other symbols
\else % if luatex or xetex
  \usepackage{unicode-math}
  \defaultfontfeatures{Scale=MatchLowercase}
  \defaultfontfeatures[\rmfamily]{Ligatures=TeX,Scale=1}
\fi
% Use upquote if available, for straight quotes in verbatim environments
\IfFileExists{upquote.sty}{\usepackage{upquote}}{}
\IfFileExists{microtype.sty}{% use microtype if available
  \usepackage[]{microtype}
  \UseMicrotypeSet[protrusion]{basicmath} % disable protrusion for tt fonts
}{}
\makeatletter
\@ifundefined{KOMAClassName}{% if non-KOMA class
  \IfFileExists{parskip.sty}{%
    \usepackage{parskip}
  }{% else
    \setlength{\parindent}{0pt}
    \setlength{\parskip}{6pt plus 2pt minus 1pt}}
}{% if KOMA class
  \KOMAoptions{parskip=half}}
\makeatother
\usepackage{xcolor}
\setlength{\emergencystretch}{3em} % prevent overfull lines
\setcounter{secnumdepth}{5}
% Make \paragraph and \subparagraph free-standing
\ifx\paragraph\undefined\else
  \let\oldparagraph\paragraph
  \renewcommand{\paragraph}[1]{\oldparagraph{#1}\mbox{}}
\fi
\ifx\subparagraph\undefined\else
  \let\oldsubparagraph\subparagraph
  \renewcommand{\subparagraph}[1]{\oldsubparagraph{#1}\mbox{}}
\fi


\providecommand{\tightlist}{%
  \setlength{\itemsep}{0pt}\setlength{\parskip}{0pt}}\usepackage{longtable,booktabs,array}
\usepackage{calc} % for calculating minipage widths
% Correct order of tables after \paragraph or \subparagraph
\usepackage{etoolbox}
\makeatletter
\patchcmd\longtable{\par}{\if@noskipsec\mbox{}\fi\par}{}{}
\makeatother
% Allow footnotes in longtable head/foot
\IfFileExists{footnotehyper.sty}{\usepackage{footnotehyper}}{\usepackage{footnote}}
\makesavenoteenv{longtable}
\usepackage{graphicx}
\makeatletter
\def\maxwidth{\ifdim\Gin@nat@width>\linewidth\linewidth\else\Gin@nat@width\fi}
\def\maxheight{\ifdim\Gin@nat@height>\textheight\textheight\else\Gin@nat@height\fi}
\makeatother
% Scale images if necessary, so that they will not overflow the page
% margins by default, and it is still possible to overwrite the defaults
% using explicit options in \includegraphics[width, height, ...]{}
\setkeys{Gin}{width=\maxwidth,height=\maxheight,keepaspectratio}
% Set default figure placement to htbp
\makeatletter
\def\fps@figure{htbp}
\makeatother
\newlength{\cslhangindent}
\setlength{\cslhangindent}{1.5em}
\newlength{\csllabelwidth}
\setlength{\csllabelwidth}{3em}
\newlength{\cslentryspacingunit} % times entry-spacing
\setlength{\cslentryspacingunit}{\parskip}
\newenvironment{CSLReferences}[2] % #1 hanging-ident, #2 entry spacing
 {% don't indent paragraphs
  \setlength{\parindent}{0pt}
  % turn on hanging indent if param 1 is 1
  \ifodd #1
  \let\oldpar\par
  \def\par{\hangindent=\cslhangindent\oldpar}
  \fi
  % set entry spacing
  \setlength{\parskip}{#2\cslentryspacingunit}
 }%
 {}
\usepackage{calc}
\newcommand{\CSLBlock}[1]{#1\hfill\break}
\newcommand{\CSLLeftMargin}[1]{\parbox[t]{\csllabelwidth}{#1}}
\newcommand{\CSLRightInline}[1]{\parbox[t]{\linewidth - \csllabelwidth}{#1}\break}
\newcommand{\CSLIndent}[1]{\hspace{\cslhangindent}#1}

\KOMAoption{captions}{tableheading}
\makeatletter
\@ifpackageloaded{tcolorbox}{}{\usepackage[many]{tcolorbox}}
\@ifpackageloaded{fontawesome5}{}{\usepackage{fontawesome5}}
\definecolor{quarto-callout-color}{HTML}{909090}
\definecolor{quarto-callout-note-color}{HTML}{0758E5}
\definecolor{quarto-callout-important-color}{HTML}{CC1914}
\definecolor{quarto-callout-warning-color}{HTML}{EB9113}
\definecolor{quarto-callout-tip-color}{HTML}{00A047}
\definecolor{quarto-callout-caution-color}{HTML}{FC5300}
\definecolor{quarto-callout-color-frame}{HTML}{acacac}
\definecolor{quarto-callout-note-color-frame}{HTML}{4582ec}
\definecolor{quarto-callout-important-color-frame}{HTML}{d9534f}
\definecolor{quarto-callout-warning-color-frame}{HTML}{f0ad4e}
\definecolor{quarto-callout-tip-color-frame}{HTML}{02b875}
\definecolor{quarto-callout-caution-color-frame}{HTML}{fd7e14}
\makeatother
\makeatletter
\makeatother
\makeatletter
\@ifpackageloaded{bookmark}{}{\usepackage{bookmark}}
\makeatother
\makeatletter
\@ifpackageloaded{caption}{}{\usepackage{caption}}
\AtBeginDocument{%
\ifdefined\contentsname
  \renewcommand*\contentsname{Table of contents}
\else
  \newcommand\contentsname{Table of contents}
\fi
\ifdefined\listfigurename
  \renewcommand*\listfigurename{List of Figures}
\else
  \newcommand\listfigurename{List of Figures}
\fi
\ifdefined\listtablename
  \renewcommand*\listtablename{List of Tables}
\else
  \newcommand\listtablename{List of Tables}
\fi
\ifdefined\figurename
  \renewcommand*\figurename{Figure}
\else
  \newcommand\figurename{Figure}
\fi
\ifdefined\tablename
  \renewcommand*\tablename{Table}
\else
  \newcommand\tablename{Table}
\fi
}
\@ifpackageloaded{float}{}{\usepackage{float}}
\floatstyle{ruled}
\@ifundefined{c@chapter}{\newfloat{codelisting}{h}{lop}}{\newfloat{codelisting}{h}{lop}[chapter]}
\floatname{codelisting}{Listing}
\newcommand*\listoflistings{\listof{codelisting}{List of Listings}}
\makeatother
\makeatletter
\@ifpackageloaded{caption}{}{\usepackage{caption}}
\@ifpackageloaded{subcaption}{}{\usepackage{subcaption}}
\makeatother
\makeatletter
\@ifpackageloaded{tcolorbox}{}{\usepackage[many]{tcolorbox}}
\makeatother
\makeatletter
\@ifundefined{shadecolor}{\definecolor{shadecolor}{rgb}{.97, .97, .97}}
\makeatother
\makeatletter
\makeatother
\ifLuaTeX
  \usepackage{selnolig}  % disable illegal ligatures
\fi
\IfFileExists{bookmark.sty}{\usepackage{bookmark}}{\usepackage{hyperref}}
\IfFileExists{xurl.sty}{\usepackage{xurl}}{} % add URL line breaks if available
\urlstyle{same} % disable monospaced font for URLs
\hypersetup{
  pdftitle={Fiscomp},
  pdfauthor={Noel Araujo Moreira},
  colorlinks=true,
  linkcolor={blue},
  filecolor={Maroon},
  citecolor={Blue},
  urlcolor={Blue},
  pdfcreator={LaTeX via pandoc}}

\title{Fiscomp}
\author{Noel Araujo Moreira}
\date{8/15/2022}

\begin{document}
\maketitle
\ifdefined\Shaded\renewenvironment{Shaded}{\begin{tcolorbox}[boxrule=0pt, borderline west={3pt}{0pt}{shadecolor}, enhanced, interior hidden, frame hidden, breakable, sharp corners]}{\end{tcolorbox}}\fi

\renewcommand*\contentsname{Table of contents}
{
\hypersetup{linkcolor=}
\setcounter{tocdepth}{2}
\tableofcontents
}
\bookmarksetup{startatroot}

\hypertarget{oluxe1}{%
\chapter*{Olá}\label{oluxe1}}
\addcontentsline{toc}{chapter}{Olá}

Texto em desenvolvimento contínuo, favor reportar os erros.

Agradecimentos a todos os meus amigos que revisaram os projetos.

Livro gerado com \url{https://quarto.org/docs/books}.

\bookmarksetup{startatroot}

\hypertarget{aquecimento}{%
\chapter{Aquecimento}\label{aquecimento}}

\textbf{Objetivo Geral}: Quebrar o gelo com a linguagem Julia.

\textbf{Objetivo Especifico}: Resolver vários exercícios de pequena
complexidade, para mostrar funcionalidades da linguagem que serão
recorrente durante o curso

\textbf{Conteúdo}: Instalação da linguagem e configuração do VS Code;
Entrada e Saída; Álgebra Linear; Estruturas de repetição; Erros
numéricos; Multiple-Dispatch

Se você já sabe Julia, ou consegue resolver os exercícios sem
dificuldades, simplesmente pule esse projeto.

\hypertarget{vetores-e-matrizes}{%
\section{Vetores e Matrizes}\label{vetores-e-matrizes}}

É muito difícil você fazer Física sem usar vetores ou matrizes, por isso
você precisa dominar operações básicas de Algebra Linear.

\begin{enumerate}
\def\labelenumi{\arabic{enumi}.}
\item
  Dados 3 vetores \(\vec{a} = (1,-1,1)\), \(\vec{b} = (-3,1,5)\) e
  \(\vec{c} = (4,-7,3)\) \footnote{Valores copiados de \cite{p1_ex2}} :

  \begin{enumerate}
  \def\labelenumii{\alph{enumii}.}
  \item
    Calcule a norma (valor absoluto Euclidiano) de cada vetor.
  \item
    Calcule o produto escalar e vetorial entre todas as combinações de
    \(\vec{a},\vec{b}\) e \(\vec{c}\).
  \item
    Calcule o ângulo entre os vetores \(\vec{ab}\) e \(\vec{ac}\).
  \item
    Crie a matriz \(A = [\vec{a}\; \vec{b}\; \vec{c}]\) e calcule seu
    determinante.
  \end{enumerate}
\end{enumerate}

\hypertarget{leitura-e-escrita}{%
\section{Leitura e Escrita}\label{leitura-e-escrita}}

Outra tarefa recorrente em programação é leitura de dados em arquivo
para posterior análise.

\begin{enumerate}
\def\labelenumi{\arabic{enumi}.}
\setcounter{enumi}{1}
\item
  Voltaremos aos tempos de ensino médio, e vamos calcular tempo de queda
  livre \(t_q = \sqrt{2h/g}\), porém não apenas na Terra.

  \begin{enumerate}
  \def\labelenumii{\alph{enumii}.}
  \item
    Pesquise o valor da aceleração da gravidade, \(g\), em diferentes
    planetas do sistema solar. Salve seus resultados em um arquivo
    \texttt{.txt}, cuja primeira coluna é o nome do planeta, e na
    segunda o valor de \(g\). Separe os dados com uma vírgula.
  \item
    Use o pacote \texttt{DelimitedFiles.jl} para ler os dados em uma
    matriz.
  \item
    Para cada planeta, calcule quanto tempo um objeto demora para
    atingir o solo, dado que a altura inicial é \(h=120 m\).
  \item
    Por meio de interpolação de texto, exiba seus resultados como:
    \texttt{"{[}planeta{]}:\ tempo\ de\ queda\ é\ {[}tempo{]}\ segundos"}.
  \item
    Use funções padrões de Julia, e descubra qual o planeta que demora
    \emph{mais} e \emph{menos} tempo para a queda acontecer.
  \end{enumerate}
\end{enumerate}

\hypertarget{estrutura-de-repetiuxe7uxe3o}{%
\section{Estrutura de Repetição}\label{estrutura-de-repetiuxe7uxe3o}}

\begin{enumerate}
\def\labelenumi{\arabic{enumi}.}
\setcounter{enumi}{2}
\item
  Ignorando um gama gigantesca de fatores, vamos assumir, o cenário
  irrealístico, de que a temperatura anual de uma cidade na região
  Sudeste do Brasil é descrito por uma função cosseno, cuja miníma anual
  é 5°, e máxima anual é 35°, acontecendo no primeiro dia do Verão, dia
  21 de dezembro (que corresponde ao 355° dia do ano).

  \begin{enumerate}
  \def\labelenumii{\alph{enumii}.}
  \item
    Crie uma função que simule o comportamento de temperatura anual.
  \item
    Vamos trabalhar as datas como índices. Crie um vetor com todos os
    números de 1 até 365.
  \item
    Calcule a temperatura anual usando \texttt{for}, \texttt{map} e o
    operador \texttt{broadcasting}, representado por um ponto
    \texttt{"."}
  \item
    Crie uma figura com seu resultado, usando o pacote
    \texttt{Plots.jl}.
  \end{enumerate}
\end{enumerate}

\begin{tcolorbox}[enhanced jigsaw, rightrule=.15mm, title=\textcolor{quarto-callout-tip-color}{\faLightbulb}\hspace{0.5em}{Dica}, colframe=quarto-callout-tip-color-frame, leftrule=.75mm, bottomtitle=1mm, opacitybacktitle=0.6, coltitle=black, arc=.35mm, toptitle=1mm, opacityback=0, bottomrule=.15mm, toprule=.15mm, titlerule=0mm, breakable, colbacktitle=quarto-callout-tip-color!10!white, left=2mm, colback=white]
Procure por \textbf{Trig word problem: modeling annual temperature} na
Internet.
\end{tcolorbox}

\hypertarget{trigonometria}{%
\section{Trigonometria}\label{trigonometria}}

Computadores são máquinas incríveis para fazer contas, mas ainda sim
possuem limites. Vamos testar um deles com o cálculo da função seno.

\begin{enumerate}
\def\labelenumi{\arabic{enumi}.}
\setcounter{enumi}{3}
\tightlist
\item
  Crie um programa para estimar o valor de \emph{sen(x)} através de sua
  expansão de Taylor truncada em \(N\) elementos: \[
      sin(x) = \sum_{n=0}^{N} (-1)^n \frac{x^{2n+1}}{(2n+1)!}
   \]
\end{enumerate}

\begin{enumerate}
\def\labelenumi{\alph{enumi}.}
\item
  Escolha um valor de \(x\) e calcule a série para \(N\) diferentes.
  Quanto maior \(N\), melhor será seus resultado, e nós desejamos
  verificar isso exibindo os erro absoluto em um plot de escala
  logarítmica. Use a biblioteca \texttt{Plots.jl} para isso.
\item
  Para quais valores de \(N\) você não consegue mais realizar a
  somatória ? Por quê?
\end{enumerate}

\begin{tcolorbox}[enhanced jigsaw, rightrule=.15mm, title=\textcolor{quarto-callout-note-color}{\faInfo}\hspace{0.5em}{Nota}, colframe=quarto-callout-note-color-frame, leftrule=.75mm, bottomtitle=1mm, opacitybacktitle=0.6, coltitle=black, arc=.35mm, toptitle=1mm, opacityback=0, bottomrule=.15mm, toprule=.15mm, titlerule=0mm, breakable, colbacktitle=quarto-callout-note-color!10!white, left=2mm, colback=white]
erro absoluto = \(|y_{estimado} - y_{exato}|\).

Use a função \texttt{sin(x)} que já vem por padrão em Julia para obter o
valor exato.
\end{tcolorbox}

\hypertarget{resistores-em-suxe9rie}{%
\section{Resistores em Série}\label{resistores-em-suxe9rie}}

Será inevitável que você fará uso de pacotes prontos. E é comum em Julia
a criação de novos tipos de dados - \texttt{tipos} no sentido
\texttt{Int} ou \texttt{Float}. Esse recurso permite que você desenvolva
códigos mais inteligível e tudo começa com \texttt{struct}. Nesse
exercício criaremos um tipo de dado para ajudar a calcular resistências
em série de circuitos elétricos.

\begin{enumerate}
\def\labelenumi{\arabic{enumi}.}
\setcounter{enumi}{4}
\tightlist
\item
  Defina um \texttt{struct}\footnote{É esperado que você busque na
    internet por exemplos mínimos de como trabalhar com \texttt{struct}.}
  chamado \texttt{resistor} com 2 campos do tipo \texttt{Number}:
  Resistência, \(R\) e Corrente, \(i\).
\end{enumerate}

\begin{enumerate}
\def\labelenumi{\alph{enumi}.}
\tightlist
\item
  Instancie 2 resistores em série, com \(i = 4A\), e \(R_1 = 5\Omega\) e
  \(R_2 = 20\Omega\).
\item
  Crie uma função que receba o tipo de entrada \texttt{resistor} e
  retorne a tensão \(U = R\cdot i\).
\item
  Crie um novo \emph{método} que some 2 resistências em série e retorne
  um novo objeto do tipo \texttt{resistor}.
\end{enumerate}

\bookmarksetup{startatroot}

\hypertarget{references}{%
\chapter*{References}\label{references}}
\addcontentsline{toc}{chapter}{References}

\hypertarget{refs}{}
\begin{CSLReferences}{0}{0}
\end{CSLReferences}



\end{document}
